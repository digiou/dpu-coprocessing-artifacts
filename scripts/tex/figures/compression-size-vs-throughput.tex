\begin{tikzpicture}

    % Manually list the devices that appear in your CSV
    % A macro that returns a more readable name given a device key
    \newcommand{\devicelabel}[1]{%
        \ifthenelse{\equal{#1}{bf2}}{BF2 HW}{%
        \ifthenelse{\equal{#1}{bf3-dpu}}{BF3 ARM}{%
        \ifthenelse{\equal{#1}{bf2-host}}{BF2 Host}{%
        \ifthenelse{\equal{#1}{bf2-dpu}}{BF2 ARM}{%
        \ifthenelse{\equal{#1}{bf3-host}}{BF3 Host}{#1}%
        }}}}%
    }

    \begin{axis}[
        width=\columnwidth,
        height=1.8cm, % Keep the height as required
        scale only axis,
        xlabel=Input Size (bytes),
        ylabel=Tput (MiB/s),
        xlabel style={yshift=3pt},   % move X label closer to the axis
        ymode=log,
        ymax=10000,
        xmode=log,
        legend style={
            legend columns=3,
            draw=none,
            fill=none,
            anchor=north east,
            at={(0.91,1.475)}
        },
        ymajorgrids=true,
        xmajorgrids=true,
        style={xshift=0pt,yshift=0pt},
    ]

    % Read the CSV table into a pgfplots data structure.
    \pgfplotstableread[col sep=comma]{figures/results/comp-dflt-size-vs-throughput.csv}\datatable

    % Plot each device/hardware column as a separate line.
    \addplot table[x=size, y=bf2]{\datatable};
    \addlegendentry{\devicelabel{bf2}}

    \addplot table[x=size, y=bf2-dpu]{\datatable};
    \addlegendentry{\devicelabel{bf2-dpu}}

    \addplot[dashed, mark=*, mark options={solid}, color=tol_olive] table[x=size, y=bf3-dpu]{\datatable};
    \addlegendentry{\devicelabel{bf3-dpu}}

    \addplot[dotted, mark=asterisk, mark options={solid}, color=black] table[x=size, y=bf2-host]{\datatable};
    \addlegendentry{\devicelabel{bf2-host}}

    \addplot[dotted, mark=diamond*, mark options={solid}, color=blue] table[x=size, y=bf3-host]{\datatable};
    \addlegendentry{\devicelabel{bf3-host}}

    % Now add a line for the ASIC performance ceiling:
    \addplot table[x=size, y=bf2-asic]{\datatable};
    \addlegendentry{BF2 ASIC}

    \end{axis}
\end{tikzpicture}
    